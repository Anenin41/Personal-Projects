% File containing Exercise 2a of CFD - Practical 4 %
% Author: Konstantine Garas
% E-mail: kgaras041@gmail.com // k.gkaras@student.rug.nl
% Created: Tue 19 Nov 2024 @ 17:33:57 +0100
% Modified: Wed 20 Nov 2024 @ 12:50:39 +0100

\section{Exercise 2}
\label{sec: exercise2}

\subsection{(a).}
In this question, I am tasked with analyzing the stability of the explicit-method A scheme, on a non-uniform grid with constant stretch rate \( \frac{h_{+}}{h_{-}} = 0.95 \).

\subsubsection{1).}
Implementing positive coefficient analysis on the \( L, D, R \) coefficients of the numerical scheme, I yield a set of inequalities that \textbf{all} must be satisfied in order for the algorithm to be numerically stable. The analysis below uses the short-hand notation introduced in sub-section \ref{subsec: explicit_method_a}. 

\begin{equation} \label{eq: condition1}
	L \geq 0 \implies \frac{u \delta t}{s(i)} + \frac{2k \delta t}{h(i-1) s(i)} \geq 0
\end{equation}
equation \ref{eq: condition1} is satisfied automatically because all of the terms are positive by construction.

\begin{align*}
	D \geq 0 &\implies \\
	1 - \frac{2k \delta t}{p(i)} &\implies \\
	2k \delta t \leq p(i) &= h(i-1) \cdot h(i)
\end{align*}

Because the stretch rate of the grid is constant, I have a constant relation between two consecutive grid distances \( h(i-1) = \frac{h(i)}{0.95} \). By plugging in this equation to the inequality above, I easily get the final form of the second condition.

\begin{equation} \label{eq: condition2}
	2k \delta t \leq \frac{\left( h(i) \right)^2}{0.95} 
\end{equation}

The last condition reads as follows:

\begin{align*}
	R \geq 0 &\implies \\
	\frac{2k \delta t}{h(i) s(i)} - \frac{u \delta t}{s(i)} \geq 0 &\implies \\
	\frac{2k \delta t}{h(i)} \geq u &\implies \\
	\frac{u h(i)}{k} \leq 2 \numberthis \label{eq: condition3}
\end{align*}

Condition \ref{eq: condition3} resembles a modified grid P\`{e}clet condition for the non-uniform grid case. However this is where the similarities end. To be more specific, since the grid is non-uniform, \( h \) here is a vector, or more precisely an array, that holds the results of \( \Delta x = x_i - x_{i-1} \), where \( x_i, x_{i-1} \) are two consecutive grid points. In addition, one might also observe that because the grid is generated in such a way that it gets finer towards the right boundary of the computational domain, the \( h \) array contains the terms of a strictly decreasing sequence of real numbers.

\[
	h_1 > h_2 > h_3 > \dots > h_{n}
\]

By this observation, and because \( h(i) \) changes with each iteration, an equivalent set of conditions can be formulated by using the minimum value of the \( h \) array. Thus, the inequalities now become:

\[
	2k \delta t \leq \frac{(\min{h})^2}{0.95} \, \text{ and } \, \frac{u \cdot \min{h}}{k} \leq 2
\]

In conclusion, I also have to check if these conditions are confirmed to be true by the program. To this end, I have \( \min{h} = 0.0046, k = 0.01, \delta t = 0.001, U = 1 \). As such, the numerical results are the following:

\[
	2.0e-05 \leq 2.2649e-05 \, \text{ and } \, 0.4639 \leq 2
\]
which are both true, meaning that the numerical scheme for these parameters is stable.

\subsubsection{2).}
The set of conditions that I derived from the non-uniform and explicit case is indeed worse that the ones drawn from the uniform setting. This is because I have to take into consideration the stretch rate, in addition to all the other factors that contribute to the stability of the method. The following table confirms this simple observation.

\begin{table}[h!]
	\begin{center}
		\begin{tabular}{ |c|c|}
			\hline
			\textbf{Uniform Case} & \textbf{Non-uniform Case} \\
			\hline 
			Thickness of the grid	& Thickness of the grid \\
			\hline	
			Time step				& Time step				\\
			\hline
			Velocity \& Diffusion coefficient & Velocity \& Diffusion coefficient \\
			\hline
			-						& \textbf{Stretch rate r} \\
			\hline
		\end{tabular}
		\caption{Factors that contribute to the stability of the numerical scheme.}
	\end{center}
\end{table}
