% File containing Exercise 3c of CFD - Practical 4 %
% Author: Konstantine Garas
% E-mail: kgaras041@gmail.com // k.gkaras@student.rug.nl
% Created: Tue 19 Nov 2024 @ 21:22:14 +0100
% Modified: Wed 20 Nov 2024 @ 13:03:35 +0100

\subsection{(c).}
According to the handout, for values of \( \eta < 0.5 \), the numerical waves travel slower than the wave of the analytic solution. In contrast, for \( \eta > 0.5 \) the numerical waves travel faster than the wave of the analytic solution. This leads us to choose \( \eta = 0.5 \), because such a choice eliminates the dispersion effect in the profile of the numerical solution.

However, as is shown in \ref{subsec: damping}, numerical damping is not eliminated with \( \eta = 0.5 \). Damping in this case influences the numerical solution in such a way, that as the time iteration goes on, the amplitude of the wave decreases.

One can also calculate the margin of the decrease on paper, and can check whether these results are right or wrong in the profile of the numerical solution. I start this analysis by noting down the constants of the problem.

\[
	\Phi(t,x) = 2 + \sin\left(6 \pi [x - U\cdot t]\right)
\]
where, \( U = 1, \delta t = 0.0051, h = 0.0101, N = 100 \). The total time of the simulation is \( T = 3 \) (in seconds), and the time snapshot of \( t = 1 \) is considered. By plugging in these values, the function in question becomes:

\[
	\Phi(1, x)) = 2 + \sin \left(6 \pi (x - 1) \right)
\]
with \( \omega = 6 \pi \). This value of \( \omega \), yields the angle of the Fourier amplification factor \( \theta = \omega \cdot h = 0.0606 \pi \). The Fourier amplification factor is quite straightforward to calculate in this case, and it is given by the equation

\[
	g(\theta) = 1 - \eta + \eta \cos(\theta) - \eta \sin(\theta) \cdot i
\]
in which \( \mathfrak{Re} = 1 - \eta + \eta \cos(\theta) \) and \( \mathfrak{Im} = - \eta \sin(\theta) \). By taking the modulus of this complex function, I can check if the method is Fourier stable. Moreover, I am going to need this value later to calculate the cumulative damping factor of the numerical solution.

\[
	\begin{cases}
		\mathfrak{Re} = 1 - 0.5 - 0.5 \cdot \cos(0.0606\pi) = 0.9910 \\
		\mathfrak{Im} = -0.5 \cdot \sin(0.0606\pi) = -0.0946 
	\end{cases} 
	\implies |g(\theta)| = \sqrt{\mathfrak{Re}^2 + \mathfrak{Im}^2} = 0.9955
\]

The time snapshot is defined as \(N \times \delta t\). Because the number of iterations until I reach \(t = 1\) is unknown, I am going to calculate it, because I will use it later as well.

\[
	t = N \cdot \delta t \implies N = 198
\]

So, the algorithm needs to perform \( 198 \) time iterations until it reaches the time snapshot of 1 second. As such, by combining all of these results, the margin of the numerical damping is calculated as follows.

\[
	\left( 1 - |g(\theta)|^{198} \right) \approx 0.5906
\]

So, the change in amplitude must be somewhere near the margin of \( 0.4 \) at \(t = 1\). The following figure shows that indeed this is an accurate result, since the travelling wave at \(t = 1\) is smaller by a factor of \( \approx 0.4 \).

\begin{center}
	\includegraphics[width=0.65\textwidth]{Extras/exercise3c.png}
\end{center}
