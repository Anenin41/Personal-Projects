% Modified final report from Model Reduction for LinkedIn %
% Author: Konstantine Gkaras
% E-mail: kgaras041@gmail.com // k.gkaras@student.rug.nl
% Created: Sat 19 Oct 2024 @ 00:30:12 +0200
% Modified: Fri 21 Feb 2025 @ 20:37:52 +0100

\documentclass{article}

%%%%%%%%%%%%%%%%%%%%%%%%%%%%%%%%%%%%%%%%%%%%%%%%%%%%%%%%%%%%%%%%%%%%
%%-----------------------PAGE SETTINGS----------------------------%%
%%%%%%%%%%%%%%%%%%%%%%%%%%%%%%%%%%%%%%%%%%%%%%%%%%%%%%%%%%%%%%%%%%%%
\usepackage[utf8]{inputenc}
\usepackage[margin=0.01cm]{geometry}

%%%%%%%%%%%%%%%%%%%%%%%%%%%%%%%%%%%%%%%%%%%%%%%%%%%%%%%%%%%%%%%%%%%%
%%--------------------------PREAMBLE------------------------------%%
%%%%%%%%%%%%%%%%%%%%%%%%%%%%%%%%%%%%%%%%%%%%%%%%%%%%%%%%%%%%%%%%%%%%
\usepackage{pdfpages}                           % Insert PDF into file without breaking margins and print output.
\usepackage{anyfontsize}                        % Use any font size.
\usepackage{setspace}                           % Customize paragraph spacing.
\usepackage{mathtools}                          % Math typing.
\usepackage{cancel}                             % Strikethrough text.
\usepackage{float}                              % Force figure on position.
\usepackage[hidelinks]{hyperref}	            % Links.
\usepackage{amsmath}                            % Math typing.
\usepackage{inputenc}                           % Special characters.
\usepackage{amsfonts}                           % Fontpack.
\usepackage{graphicx}                           % Graphics.
\usepackage{enumitem}                           % Special numerisation on lists.
\usepackage{amsthm}                             % Math.
\usepackage{xcolor}                             % Colors.
\usepackage{lipsum}                             % Dummy text.
\usepackage{url}								% Url's.
\usepackage{lmodern}							% Latin Modern for Computer Fonts.
\usepackage{textcomp}							% Special symbols like degrees or euro.
\usepackage{listings}							% Input code directly from file or path.
\usepackage{pdfpages}							% Include pdf documents in the output file.
\usepackage{amssymb}							% Math.
\usepackage{tikz}								% Drawing plots.

%%%%%%%%%%%%%%%%%%%%%%%%%%%%%%%%%%%%%%%%%%%%%%%%%%%%%%%%%%%%%%%%%%%%
%%-----------------------CUSTOM COMMANDS--------------------------%%
%%%%%%%%%%%%%%%%%%%%%%%%%%%%%%%%%%%%%%%%%%%%%%%%%%%%%%%%%%%%%%%%%%%%
\newcommand{\textBF}[1]{%
    \pdfliteral direct {2 Tr 1 w}               %the second factor is the boldness
     #1%
    \pdfliteral direct {0 Tr 0 w}               %
}

\newcommand{\textDF}[1]{%
    \pdfliteral direct {2 Tr 0.2 w}             %the second factor is the boldness
     #1%
    \pdfliteral direct {0 Tr 0 w}               %
}

\definecolor{codegreen}{rgb}{0,0.6,0}
\definecolor{codegray}{rgb}{0.5,0.5,0.5}
\definecolor{codepurple}{rgb}{0.58,0,0.82}
\definecolor{backcolour}{rgb}{0.95,0.95,0.92}

\lstdefinestyle{mystyle}{
    backgroundcolor=\color{backcolour},   
    commentstyle=\color{codegreen},
    keywordstyle=\color{magenta},
    numberstyle=\tiny\color{codegray},
    stringstyle=\color{codepurple},
    basicstyle=\ttfamily\footnotesize,
    breakatwhitespace=false,         
    breaklines=true,                 
    captionpos=b,                    
    keepspaces=true,                 
    numbers=left,                    
    numbersep=5pt,                  
    showspaces=false,                
    showstringspaces=false,
    showtabs=false,                  
    tabsize=2
}

\lstset{style=mystyle}

% Define the abs() function
\newcommand{\abs}[1]{\left\lvert #1 \right\rvert}

% Define the norm() function
\newcommand{\norm}[1]{\left\lVert #1 \right\rVert}

% Define the inner product using angle brackets
\newcommand{\inner}[2]{\left\lange #1, #2 \right\rangle}

% Define the argmin operator
\DeclareMathOperator*{\argmin}{argmin}

% Define the dist operator
\DeclareMathOperator*{\dist}{dist}

% Define the resolvent operator
\DeclareMathOperator{\res}{Res}

% Define the proximal operator
\DeclareMathOperator*{\prox}{prox}

% Redefine the proof environment to be completely silent
\renewenvironment{proof}%
{\noindent}%
{\hfill$\square$\par}

% Start section counter from 0
%\setcounter{section}{-1}

% Define the Remark environment
\theoremstyle{remark}
\newtheorem*{remark}{Remark}

%%%%%%%%%%%%%%%%%%%%%%%%%%%%%%%%%%%%%%%%%%%%%%%%%%%%%%%%%%%%%%%%%%%%
%%-----------------------DOCUMENT BEGIN---------------------------%%
%%%%%%%%%%%%%%%%%%%%%%%%%%%%%%%%%%%%%%%%%%%%%%%%%%%%%%%%%%%%%%%%%%%%
\begin{document}

\begin{titlepage}
\thispagestyle{empty}
\title{
\includegraphics[width=19cm]{Extras/Mathlogo.PNG} \\
\vspace{3cm}
\begingroup
\setstretch{4}\fontsize{25}{10}\selectfont\fontdimen2\font=0.8ex
% Project Title  %
% If you want the title centered, leave the command as it is, 
% otherwise delete the \center{} command.
\parbox{15cm}{\center{\textBF{Enhancing Black-Scholes Option Pricing: A Method of Moments Approach}}}
\endgroup}
\date{}
\maketitle
\vspace{-1.5cm}
\hspace{3cm}\parbox[b][15cm][b]{15cm}{\textDF{\large\setstretch{1.5}
Report under the course of Model Reduction for Partial Differential Equations \\
\date{October 24, 2024}\\
Author: Konstantinos Gkaras\\ %NAME
Contact Information: \href{mailto: kgaras041@gmail.com}{kgaras041@gmail.com}
}}
\end{titlepage}
\newpage
% Define new page geometry because the default was changed for title page.
\newgeometry{top=1in,bottom=1in,right=1in,left=1in}

% Input the abstract into the document.
% Abstract for modified final report of Model Reduction %
% Author: Konstantine Garas
% E-mail: kgaras041@gmail.com // k.gkaras@student.rug.nl
% Created: Wed 19 Feb 2025 @ 17:09:23 +0100
% Modified: Wed 19 Feb 2025 @ 18:32:13 +0100

\begin{abstract}
	    The Black-Scholes model remains a cornerstone of financial derivatives pricing, but it assumes normality in stock returns, ignoring real-world asymmetries. This report applies a correction to the original model via a combination of the Gram-Charlier expansion and the Method of Moments (MoM). By incorporating skewness and excess kurtosis to the derivative pricing scheme, accuracy is improved, reducing pricing errors and yielding more realistic option valuations compared to the standard Black-Scholes model.
\end{abstract}


% Table of contents
%\tableofcontents

% Input document sub-files, following the divide and conquer strategy.
% Introduction to the modified final report about MoM in Finance %
% Author: Konstantine Garas
% E-mail: kgaras041@gmail.com // k.gkaras@student.rug.nl
% Created: Wed 19 Feb 2025 @ 17:27:01 +0100
% Modified: Wed 19 Feb 2025 @ 18:57:14 +0100

\section{Introduction}
\label{sec: intro}

The Black-Scholes model \cite{black1973pricing} is one of the most widely used mathematical frameworks for pricing European options. By assuming that asset prices follow a geometric Brownian motion, and that asset returns are normally distributed, the model provides a closed-form solution for option valuation. However, real-life market data are asymmetric, in the sense that they often exhibit skewness and excess kurtosis, not assumed by the original model.

The presence of these statistical characteristics means that the Black-Scholes model misprices options. Capitalizing on this feature, this report enhances the pricing scheme by introducing corrections using the Method of Moments and the Gram-Charlier expansion \cite{wallace1958asymptotic}. By incorporating higher order moments to the probability density function, an adjustment is made to the model to account for non-normal market behavior, overall resulting in a more accurate representation of option prices.

			% Introduction
% Section about the ansatz and the Gram-Charlier A series %
% Author: Konstantine Garas
% E-mail: kgaras041@gmail.com // k.gkaras@student.rug.nl
% Created: Wed 19 Feb 2025 @ 17:10:33 +0100
% Modified: Fri 21 Feb 2025 @ 18:22:38 +0100

\section{Building the Ansatz}
\label{sec: ansatz}

This section introduces the mathematical tools that will be needed in modifying the Black-Scholes model. The discussion starts by introducing the key concepts behind the model.

Let $\mathcal{X}(t)$ be a real-valued random variable that depends on time. This mathematical construct (that is also known as a stochastic process) is often used to model a variety of physical or artificial phenomena that change randomly over time. Due to their nature, stochastic processes are commonly used in the stock market as a means to analyze the future behaviour of stock prices or other complex financial assets. 

The probability of a time-dependent random variable to take a value less than, or equal to \( x \), is given by its cumulative distribution function (CDF). This function, is defined as follows.

\[
F_{\mathcal{X}(t)}(x,t) = \mathbb{P}(\mathcal{X}(t) \leq x)
\]

From this CDF, one can also produce the probability density function (PDF) of the random variable by applying the derivative operator.

\[
f_{\mathcal{X}(t)}(x,t) = \frac{\partial F_{\mathcal{X}(t)}(x,t)}{\partial x}
\]

A direct result from probability theory is that the PDF basically expresses the probability of the time-dependent random variable to take values within the range of $(a,b]$. Mathematically this is formulated by an integral.

\[
\mathbb{P}(a < \mathcal{X}(t) \leq b) = F_{\mathcal{X}(t)} (b,t) - F_{\mathcal{X}(t)}(a,t) = \int_{a}^{b} f_{\mathcal{X}(t)} (x,t) dx
\]

In this setting, the CDF and PDF functions are unknown since we don't have the information of how the financial asset will evolve over time. As such, the is the need to approximate them in some way. This is done by using the \textbf{Gram-Charlier A series} described in \cite{wallace1958asymptotic}. Before the introduction of the Method of Moments however, some mathematical preliminaries are required.

\subsection{Preliminaries}
The \textbf{Gram-Charlier A series} uses the \textit{characteristic function} of the distribution of a random variable and expands it based on its \textit{cumulants}, with both concepts being thoroughly analyzed in \cite{blitzstein2019introduction}. Generally, the characteristic function is given by the following form.

\[
\phi_{\mathcal{X}(t)} (t) = \text{E}[e^{it \mathcal{X}(t)}]
\]

It is worthwhile to mention here, that what statisticians call the characteristic function of a distribution, is actually nothing more that its \textit{Fourier Transform}. Fourier Transformations will be used extensively later, to derive the final formula of the Gram-Charlier expansion.

In addition, the expansion also uses the cumulants of a random variable. Cumulants are mathematical terms that are closely related to the moments of a distribution function, and like moments, they are generated by a \textit{generation function}.

\[
	K_{\tau}(t) = \log(\text{E}[e^{t \mathcal{X}(\tau)}])
\]

\begin{remark}
	In the context that is described in this report, $\mathcal{X}(t)$ is a stochastic process. There are many different ways to classify such mathematical objects, which actually depend on the time set, the index set or dependence among the random variables. To be more specific, depending on the cardinality of the time set, the stochastic processes are categorized and studied in a different manner. 
	In the case of this report, such intricate mathematical theory is not needed, since the time set is small and countable. For example under Section \ref{sec: application}, the time of maturity of the European call option will be \(T = 2\). So, the time set \(\mathcal{T} = \{0,1,2\} \), where $t=0$ denotes the current year, and $t=2$ is for example 2 years into the future.
\end{remark}

Because $\mathcal{X}_t$ is completely unknown, it is not possible to calculate its characteristic functions or its cumulants, due to lack of information. However, through the properties of those functions, one can build an approximation to the unknown CDF and PDF, which will later be used in the Gram-Charlier expansion.

By stabilizing the time step of the stochastic process and expanding the cumulant-generating function using the MacLaurin series around zero, the following power series emerges.

\[
K(t) = \sum_{n=1}^{\infty} \kappa_{n} \frac{t^n}{n!}
\]

Using this equation, the $n$-th cumulant of $\mathcal{X}_t$ is given by derivation of the MacLaurin series $n$ times at $t=0$, since \( K(t) \) is a polynomial function. 

It has already been mentioned that the cumulants are closely related to the moments of a distribution function, in the sense that they allow statisticians to study core properties of said distribution. More specifically, the first 4 cumulants, which are linked to the most important moments of a distribution, are defined by the following expressions.


\begin{align*}
	\kappa_{1} (\mathcal{X}_t) &= \text{E}[\mathcal{X}_t]	\\
	\kappa_{2} (\mathcal{X}_t) &= \text{Var}[\mathcal{X}_t] \\
	\kappa_{3} (\mathcal{X}_t) &= \text{E}[(\mathcal{X}_t - \text{E}[\mathcal{X}_t])^3]\\
	\kappa_{4} (\mathcal{X}_t) &= \text{E}[(\mathcal{X}_t - \text{E}[\mathcal{X}_t])^4]- 3 ( \text{E}[(\mathcal{X}_t - \text{E}[\mathcal{X}_t])^2])^2
\end{align*}

Notice that the 1st cumulant is the 1st raw moment of the random variable, or \textit{mean}. Moreover, the 2nd cumulant is the 2nd central moment of the random variable, or the \textit{variance}. The 3rd cumulant is the 3rd central moment, and the 4th cumulant is the 4th central moment, minus 3 times the square of the second central moment.

In a probability distribution, the major properties one is interested in is the mean, the variance, the skewness and the kurtosis. These properties are connected with the raw, central, and standardized moments respectively, and as a direct result, they are also connected with the cumulants.

\begin{remark}
	The $n$-th standardized moment of a random variable is nothing more that the $n$-th central moment divided by $\sigma^n$.
\end{remark}

By combining the equations for the first 4 cumulants and the result from the remark above, one gets relations for the skewness and the kurtosis.


\begin{gather*}
	\text{Skew}[\mathcal{X}_t] = \frac{\kappa_3}{\sigma^3} \\
	\text{Kurt}[\mathcal{X}_t] = \frac{\kappa_4 + 3 \sigma^2}{\sigma^4}
\end{gather*}


In probability theory, skewness and kurtosis are the 3rd and 4th standardized moments of a random variable, respectively. Skewness measures the asymmetry of a distribution, while kurtosis characterizes the tail of the distribution (with higher than 3 kurtosis meaning for example a more "fat" tail). Introductory books in probability theory, like \cite{blitzstein2019introduction}, cover the subject of moments thoroughly.

\subsection{Gram-Charlier Expansion}
The idea of the Gram-Charlier A series, is the asymptotic expansion of an unknown distribution function, based on its cumulants, and a known distribution that serves as basis for the expansion.

To this end, let \( \mathcal{X}(t) \) the stochastic process in question, evaluated at constant time, i.e. \( \mathcal{X}_t \). For the basis of the asymptotic expansion I will use the Normal distribution, which is also one of the core assumptions of the Black-Scholes model. As such, let $\hat{f}$ and $\hat{\psi}$ be the characteristic functions of $\mathcal{X}_t$ and $\mathcal{N}(\mu, \sigma^2)$ respectively. Consider $\gamma_r$ the cumulants of the Normal distribution. By once again using the MacLaurin series around zero, the following power series for the Normal characteristic function emerges.

\[
\hat{\psi} = \exp \left( \sum_{r=1}^{\infty} \gamma_r \frac{(it)^r}{r!} \right)
\]

By properties of the exponential function, one can connect the expansion of $\hat{f}$ and $\hat{\psi}$ quite easily.

\[
\hat{f} = \exp \left[ \sum_{r=1}^{\infty} (\kappa_{r} - \gamma_{r}) \frac{(it)^r}{r!} \right] \hat{\psi}(t) \tag{1}
\]

It is worthwhile to notice, that the mathematical construct statisticians call the characteristic function, it is actually nothing more than the \textit{Fourier Transformation} of the probability density function. More specifically, the Fourier Transformation is defined as follows.

\[ 
\mathcal{F} \{ f(x) \} = \int_{-\infty}^{\infty} f(x) e^{-itx} dx
\]

This transformation has very useful mathematical properties. All of the methods that are used below use these widely established characteristics of the Fourier operator, and for reference, they are thoroughly analysed in \cite{herman2015introduction}, on Chapter 9.5. Apart from this, characteristic functions and Fourier transformations are crucial in obtaining the final equation of the \textbf{Gram-Charlier A} series. To this end, I start with the derivative property.

\[
\mathcal{F} \left\{ \frac{d^r}{dx^r} f(x) \right\} = (it)^r \hat{f}(x) \tag{2}
\]

Moreover, Fourier Transforms are bijective operators, meaning that (2) can be inverted.

\begin{gather*}
	\mathcal{F}^{-1} \left\{ \mathcal{F} \left\{ \frac{d^r}{dx^r} \right\} f(x) \right\} = \mathcal{F}^{-1} \{(it)^r \hat{f}(t)\} \implies \\
	\frac{d^r}{dx^r} f(x) = \mathcal{F}^{-1} \{ (it)^r \hat{f}(t) \}
\end{gather*}

By using the result from above in combination with equation (1), a relation including the original characteristic function of the Normal distribution is generated.

\[
	\hat{\psi}(t) = \int_{-\infty}^{\infty} \psi(x) e^{-itx} dx
\]

\[
\mathcal{F} \left\{ \frac{d^r}{dx^r} \psi(x) \right\} = (it)^r \hat{\psi}(t) \tag{3}
\]

A trick is needed to proceed. First, notice simply the following equation.

\[
	\mathcal{F} \{\psi(-x)\}(t) = \int_{-\infty}^{\infty} \psi(-x) e^{-itx}dx = \int_{\infty}^{-\infty} \psi(u) e^{itu} (-du)
\]
by performing a change of variables $u = -x$, the differential also changes to $du = -dx$. However, by taking into consideration that the Fourier Transform is even symmetric (Hermitian Function) for real-valued functions, it is possible to use the time-reversal property of this operator and get equation (4).

\[
\int_{\infty}^{-\infty} \psi(u) e^{itu} d(-u) = \int_{-\infty}^{\infty} \psi(u) e^{-itu}du = \hat{\psi}(-t) \tag{4}
\]

So, $\mathcal{F}\{ \psi(-x) \}(t) = \hat{\psi}(-t)$. Now apply the differential operator $r$ times.

\[
	\frac{d^r}{dx^r} \psi(-x) = (-1)^r \frac{d^r}{d(-x)^r} \psi(-x)
\]

By plugging in this equations on (3), and by using the linearity of the Fourier Transform, an equation arises that calculates the differential of $\psi(-x)$. 

\[
	\mathcal{F} \left\{ (-1)^r \frac{d^r}{d(-x)^r} \psi(-x) \right\} = (-1)^r \mathcal{F} \left\{ \frac{d^r}{d(-x)^r} \psi(-x) \right\} \xRightarrow{(3),(4)} (-1)^r (it)^r \hat{\psi}(t)
\]

Lastly, by setting $r=0$, an equation regarding the original function arises as follows.

\begin{gather*}
	\frac{d^r}{dx^r} f(x) = \mathcal{F}^{-1} \left\{ (it)^r \hat{f} (t) \right\} \xRightarrow[]{r=0} \\
	f(x) = \mathcal{F}^{-1} \left\{ (it)^0 \hat{f}(t) \right\} \implies \\
	f(x) = \mathcal{F}^{-1} \left\{ \hat{f}(t) \right\} \tag{5}
\end{gather*}

It is now time to use all these properties to modify the original expansion. More specifically, use (5), in combination with (4) and (1) to get an expansion regarding the probability density functions, and not the characteristic functions.

\[
	f(-x) = \exp \left[ \sum_{r=1}^{\infty} \frac{\kappa_{r} - \gamma_{r}}{r!} \cdot \left( - \frac{d^r}{d(-x)^r} \right) \right] \psi(-x)
\]

By changing the variable from $-x$ to $x$, one gets the final form of the power series.

\[
	f(x) = \exp \left[ \sum_{r=1}^{\infty} \frac{\kappa_r - \gamma_r}{r!} \cdot \left( - \frac{d^r}{d(-x)^r} \right) \right] \psi(x) \tag{6}
\]

Equation number (6) expands the unknown probability density function with regards to its cumulants and a known probability density function. It is obvious that by choosing different forms of $\psi(x)$, the expansion will be different each time. It was mentioned before that the Normal Distribution will be used to expand the PDF. To this end, let $\psi(x) = \phi(x)$ the PDF of $\mathcal{N}(\mu, \sigma^2)$. Because of the properties of the Normal Distribution, this expansion can be analyzed further.

\begin{remark}
	The first and second cumulants of $\mathcal{N}(\mu, \sigma)$, are the mean and variance, and they are chosen to be equal with the mean and variance of $\mathcal{X}_t$. As such, the first two terms of the power series are zero, because $\kappa_1 = \gamma_1 = \mu$ and $\kappa_2 = \gamma_2 = \sigma^2$. 
	Moreover, all higher order moments of the Normal Distribution are zero! This leads to all the higher order cumulants \( \gamma_r \) with \(r > 2\) being zero as well. This nice observation allows us to mathematically modify the power series into an infinite sum that only contains the cumulants of $\mathcal{X}_t$.
\end{remark}

By changing the power series index from $r=0$ to $r=3$, I can discard the $\gamma_r$ term. 

\[
	f(x) = \exp \left[ \sum_{r=3}^{\infty} \frac{\kappa_r}{r!} \cdot \left( - \frac{d^r}{d(-x)^r} \right) \right] \phi(x) \tag{7}
\]

There exists a really nice recursion formula for the $r$-th derivative of the Normal PDF. This recursive law is introduced in \cite{patel1996handbook}, and connects the derivatives of the PDF with the Hermite polynomials of appropriate degree.

\[
	\frac{d^r}{dx^r} \phi(x) = \phi^{(r)}(x) = \frac{(-1)^r}{\sigma^r} \text{He}_r \left(\frac{x - \mu}{\sigma} \right) \phi(x) \tag{8}
\]
where $\text{He}_r$ is the (probabilistic) Hermite Polynomial of degree $r$.

In addition to all of these results, the cumulants can also be expressed in a combinatorial way using Bell Polynomials. To be more specific, the $n$-th order complete Bell Polynomial is generated by the following expression.

\[ 
	B_n(x_1, x_2, \dots, x_n) = \sum_{1j_1 + 2j_2 + \cdots + nj_n = n} n! \prod_{i=1}^{n} \frac{x_{i}^{j_i}}{(i!)^{j_i} j_i !}
\]
where $j_i$ are non-negative integers satisfying

\[ 
	1j_1 + 2j_2 + \cdots + nj_n = n
\]

By replacing $x_i = \kappa_i$, I can rewrite the expression above in the following way.

\[  
	B_n(\kappa_1, \dots, \kappa_n) = \sum_{1j_1 + \cdots + nj_n = n} \frac{n!}{j_1! j_2! \cdots j_n!} \left( \frac{\kappa_1}{1!} \right)^{j_1} \cdots \, \left( \frac{\kappa_n}{n!} \right)^{j_n} \tag{9}
\]

Now is the time to use the context of the application and shift away from pure mathematical theory. In truth, I am only interested in $\kappa_3$ and $\kappa_4$ since through them, I can study the skewness and kurtosis of the dataset. As such, by applying equations (8) and (9) to (7), I can further transform the ansatz.

\[
	f(x) = \phi(x) \sum_{n=0}^{\infty} \frac{1}{n! \sigma^n} B_n(0, 0, \kappa_3, \dots, \kappa_n) \text{He}_n \left( \frac{x - \mu}{\sigma} \right) \tag{10}
\]

One can also integrate equation (10) to get the CDF of the random variable in question.

\[
	F(x) = \int_{-\infty}^{x} f(u)d(u) = \Phi(x) - \phi(x) \sum_{n=3}^{\infty} \frac{1}{n! \sigma^{n-1}} B_n(0, 0, \kappa_3, \dots, \kappa_n) \text{He}_{n-1} \left( \frac{x-\mu}{\sigma} \right) \tag{11}
\]

These final two equations are the \textbf{Gram-Charlier A series}, and since I am only interested in skewness and kurtosis, I only expand the power series up to $\kappa_4$. This is easily done by taking \(j_1, j_2 = 0\) and \(j_3 = 1 \) to the Bell Polynomial or degree 3, and \(j_1, j_2, j_3 = 0, j_4 = 1\) on the Bell Polynomial of degree 4. Lastly, to make the expansion slightly more visually appealing, a change of variables is performed.

\begin{gather*}
	F(\sigma \xi + \mu) \approx \Phi(\sigma \xi + \mu) - \phi(\sigma \xi + \mu) \sum_{n=3}^{4} \frac{1}{n! \sigma^{n-1}} B_n(0, 0, \kappa_3, \kappa_4) \text{He}_{n-1} (\xi) \\
	f(\sigma \xi + \mu) \approx \phi(\sigma \xi + \mu) \left[ 1 + \frac{\kappa_3}{3! \sigma^3} \text{He}_{3}(\xi) + \frac{\kappa_4}{4!{\sigma^4}} \text{He}_4(\xi) \right]
\end{gather*}
			% Ansatz & Gram-Charlier
% Section about the Black-Scholes Model %
% Author: Konstantine Garas
% E-mail: kgaras041@gmail.com // k.gkaras@student.rug.nl
% Created: Wed 19 Feb 2025 @ 17:11:32 +0100
% Modified: Fri 21 Feb 2025 @ 19:08:56 +0100

\section{Black-Scholes Model}
\label{sec: model}
The Black-Scholes model is one of the most widely used option pricing models in Finance. It provides a closed-form solution for the pricing of European options, by assuming that asset prices follow a geometric Brownian motion with constant volatility.

However, real-world market conditions deviate from these assumptions, particularly in the presence of skewness and excess kurtosis in asset returns. In truth, the data almost never follow a strict Normal distribution, leading to the mispricing of options from the original model, due to the presence of higher order cumulants. This section tries to solve this problem, by introducing the Black-Scholes model, alongside its key components. Then, the Gram-Charlier expansion is applied to the default model, improving accuracy by accounting for deviations from normality.

\subsection{General \& Market Related Notation}
\begin{itemize}
	\item $t$ is the time in years, with $t=0$ denoting the current year and the present date.
	\item $r$ is the annualized risk-free rate, also known as the force of interest. This is generally unknown but can be calculated from complex financial models regarding the yield of treasury bonds. In general however, it is assumed to be $5\%$.
\end{itemize}

\subsection{Asset Related Notation}

\begin{itemize}
	\item $S(t)$ is the price of the underlying asset at time $t$. In this report the underlying asset is taken to be the stock of NVIDIA corporation.
	\item $\mu$ is the drift rate of $S(t)$, annualized.
\end{itemize}

\begin{remark}
	In many financial models, the stock price $S(t)$ is modelled by a time-dependent random variable, a construct that is also known as a stochastic process. Many financial models try to study the random fluctuations of stock prices, and in order to do so, almost all of them assume that the continuous stochastic process $S(t)$ follows a geometric Brownian motion.
	The theory behind this model assumes that the natural logarithm of $S(t)$ follows a Brownian motion with drift $\mu$. Because calculating results from this setup requires really intricate knowledge of probability theory, statisticians found a way to calculate the drift rate of $S(t)$ by analyzing the mean daily returns of the stock (closing price of the stock today, minus the closing price of the stock yesterday) over a period of one year.
\end{remark}

\begin{itemize}
	\item $\sigma$ is the standard deviation of a stock's returns, annualized, which is a measurement of the volatility of the stock. Mathematically this is given by a really complex formula, but can actually be approximated from market data.
\end{itemize}

\subsubsection{Volatility}
One way to approximate the volatility of the stock is by using the market data \cite{cabrera}. Start by calculating the daily returns for day $t$.
\[
	r_t = \ln \left( \frac{P_t}{P_{t-1}} \right)
\]
where $P_t, P_{t-1}$ is the closing price of the stock on day $t$ and $t-1$ respectively.

The standard deviation $\sigma$ of the daily returns is given by:
\[
	\sigma_{\text{daily}} = \sqrt{ \frac{1}{n-1} \sum_{t=1}^{n} (r_t - \bar{r})^2}
\]
where $r_t$ is the return on day $t$, $\bar{r}$ is the average return over the period of $n$ number of days.

In order to annualize the volatility, people often use the formula:
\[
	\sigma_{\text{annual}} = \sigma_{\text{daily}} \times \sqrt{252}
\]
where 252 are usually the trading days in a year. The annual volatility represents the expected standard deviation of the stock over a year.

\subsection{Option Related Notation}

\begin{itemize}
	\item $V \left( S(t),t \right)$ is the price of the option as a function of the stock price at time $t$. More specifically:
		\begin{itemize}
			\item $C \left( S(t),t \right)$ is the price of the European call option.
			\item $P \left( S(t),t \right)$ is the price of the European put option.
		\end{itemize}
	\item $T$ is the time of expiration of the option.
	\item $\tau$ is the time until maturity $T-t$, or how long until the option expires.
	\item $K$ is the strike price of the option.
\end{itemize}

\begin{remark}
	In finance an option is a \textbf{contract} which conveys to its owner the right, but not the obligation, to buy or sell a specific quantity of stock at a specified strike price on, or before a specified date.
	The strike price is a fixed value at which the owner of a stock can buy or sell the stock.
\end{remark}

As a direct observation from what options are, it is evident that option contracts can \textbf{expire}. In addition, they are categorized into \textbf{call options} contracts to buy an amount of stock, and \textbf{put options} contracts to sell an amount of stock.

\subsection{Black-Scholes Equation}
The Black-Scholes \cite{black1973pricing} equation is a parabolic partial differential equation that describes how the price of the option fluctuates in time, in regard to the stock price $S(t)$. For terminal condition (regarding the expiration date of the option), and boundary conditions, this Stochastic Differential Equation can be solved.

\[ 
	\frac{\partial V}{\partial t} + \frac{1}{2} \sigma^2 S^2 \frac{\partial^2 V}{\partial S^2} + r S \frac{\partial V}{\partial S} - rV = 0
\]

\subsubsection{Boundary Conditions}
\hspace{\parindent}If the stock price reaches zero (meaning that the company goes bankrupt), then there is no saving the company.
\[ 
	C(0,t) = 0 \,\,\,\, \forall t
\]

As the stock price grows arbitrarily over time, then the call option goes to $S-K$.
\[
	C(S(t),t) \to S-K \text{ as } S \to \infty
\]

At the time of expiration, the call option returns some profit or nothing at all, meaning that the investor can't go into dept for not exercising the option (if the prerequisites of the contract are not met, or if the contract expires).
\[
	C(S(T), T) = \max(S-K, 0)
\]

\subsubsection{Solution}
Using these boundary conditions, the solution of the Black-Scholes equation has the following form.

\begin{gather*}
	C(S(t),t) = \Phi(d_{+})S(t) - \Phi(d_{-})Ke^{-r(T-t)} \\
	d_{+} = \frac{1}{\sigma \sqrt{ T - t}} \left[ \ln\left( \frac{S(t)}{K} \right) +\left( r+\frac{\sigma^2}{2} \right)(T-t) \right] \\	
	d_{-} = d_{+} - \sigma \sqrt{ T-t }
\end{gather*}
where $\Phi(\cdot)$ is the CDF of the Normal Distribution.

By this formulation and solution, it is clear that the Black-Scholes model assumes that the returns (the profit of investment) are normally distributed. This is a nice mathematically because most probabilistic and statistic results reference in some way the Normal Distribution. However, such an assumption is mostly inaccurate.

The reason why this scenario is far-fetched is because the Normal Distribution has no skewness and kurtosis (since all higher order moments/cumulants are zero). That means that the profits of the investment always follow a symmetric distribution, with no slender or fat tail. In the opposite, real market data often showcase asymmetry and slender, or fat, tails.

As such, the Black-Scholes model underestimates option prices for highly skewed assets. Moreover, it also fails to capture the probability of extreme market movement. This is more evident for volatile stocks, meaning stocks that wildly fluctuate between values. The stock price of the NVIDIA Corporation is a prime example of these features.

In order to correct this bad behaviour, and to make a more appropriate fitting in the market data that is widely available, the Gram-Charlier expansion is introduced into the Black-Scholes model. This adjustment corrects the model and allows it to take into consideration higher order cumulants that cannot be discarded in the data, while also improving option pricing. Overall, this leads to a more accurate model, which maintains the usefulness of the original Black-Scholes as its foundation.

\begin{gather*}
	C_{\text{MoM}} = C_{\text{BS}} + \text{ Adjustments for Skewness and Kurtosis} \\
	C_{\text{MoM}} = C_{\text{BS}} \, \times \, \left[ 1 + \frac{\kappa_{3}}{3!} \text{He}_{3}(d_{+}) + \frac{\kappa_{4}}{4!} \text{He}_{4}(d_{+}) \right]
\end{gather*}
			% Black-Scholes Model
% Section about the Application of the Model to a Real-Life Example %
% Author: Konstantine Garas
% E-mail: kgaras041@gmail.com // k.gkaras@student.rug.nl
% Created: Wed 19 Feb 2025 @ 17:12:10 +0100
% Modified: Fri 21 Feb 2025 @ 20:44:50 +0100

\section{Application}
\label{sec: application}
To showcase that this new model actually performs as desired, a practical application is also introduced. Here, a direct comparison takes place between the original and adjusted Black-Scholes models, following semi-real market data that satisfy the assumptions of the previous section.

As it has been discussed in the previous section, that the core assumption the Black-Scholes model makes is normality. However, in the following tables, a comparison is being made between the original model and the Method of Moments adjusted one. A common baseline was used in the calculations. More specifically, the stock price is \( S(t) = \text{\texteuro} 137.29 \), and the time of maturity of the \textbf{call option} is 2 years, and the tables were generated using code from \cite{githubrepo}.

\begin{table}[H]
	\centering
	\begin{tabular}{|c|c|c|c|c|}
		\hline
		Strike Price (\texteuro) & Time to Maturity (Years)	& Black-Scholes (\texteuro)	& MoM Black-Scholes (\texteuro)	& Difference \\
		\hline
		45 & 2 & 96.65 & 96.65 & 0 \% \\
		50 & 2 & 92.21 & 92.21 & 0 \% \\
		60 & 2 & 83.49 & 83.49 & 0 \% \\
		\hline
	\end{tabular}
	\caption{The difference in performance between the two models, assuming normality.}
\end{table}

\begin{table}[H]
	\centering
	\begin{tabular}{|c|c|c|c|c|}
		\hline
		Strike Price (\texteuro) & Time to Maturity (Years)	& Black-Scholes (\texteuro)	& MoM Black-Scholes (\texteuro)	& Difference \\
		\hline
		45 & 2 & 96.65 & 97.28 & 0.65 \% \\
		50 & 2 & 92.21 & 93.41 & 1.30 \% \\
		60 & 2 & 83.49 & 85.92 & 2.91 \% \\
		\hline
	\end{tabular}
	\caption{The difference in performance between the two models, assuming \(\text{Skew} = -0.5, \text{Kurt} = 4 \).}
\end{table}

\begin{table}[H]
	\centering
	\begin{tabular}{|c|c|c|c|c|}
		\hline
		Strike Price (\texteuro) & Time to Maturity (Years)	& Black-Scholes (\texteuro)	& MoM Black-Scholes (\texteuro)	& Difference \\
		\hline
		45 & 2 & 96.65 & 98.55 & 1.97 \% \\
		50 & 2 & 92.21 & 95.81 & 3.90 \% \\
		60 & 2 & 83.49 & 90.78 & 8.73 \% \\
		\hline
	\end{tabular}
	\caption{The difference in performance between the two models, assuming \(\text{Skew} = -1.5, \text{Kurt} = 6 \).}
\end{table}

By observing the numerical results on the table, it is noticeable that the corrected model predicts slightly higher option prices than the default one. This is because excess skewness and kurtosis are introduced to the data. Moreover, the impact of corrections increases as skewness and kurtosis deviate further from normality, a behaviour which indicates that the standard Black-Scholes model under-estimates option prices.

Lastly, it is evident from this simple example that the difference between the values remains moderate. Such a change might not seem like much, but it is highly significant for high-value contracts. 

To conclude this report, first I would like to thank you for the time that you devoted in studying my work. In addition some direct implications can also be formulated for the corrected model.

\begin{itemize}
	\item Using the corrected model, investors, who have a much higher intuitive understanding on the mechanisms of the market, can use this model to better predict how their strategies will perform.
	\item The Method of Moments Black-Scholes model ensures option prices that reflect real world tail behaviour. It can be even further improved mathematically, by using the Log-Normal distribution as the basis of the Gram-Charlier expansion, a distribution which has higher order moments, and would actively correct all of the statistically important terms of the approximation.
	\item Adjusting the model for real-world distributions helps maintain fair pricing in the market. I wouldn't be surprised if journals that focus on finance use derivations of the Black-Scholes model to correctly fit their data into predictive strategies.
\end{itemize}

		% Application

% Manage references and plug them into the table of contents.
\addcontentsline{toc}{section}{References}
\bibliographystyle{unsrt}
\bibliography{/home/anenin/Documents/Projects/LaTeX/templates/article/references.bib}

\end{document}
